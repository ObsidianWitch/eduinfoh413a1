\section{Exercise 1.2}

\subsection{Implementation}
The second exercise reuses most of what has been developed for the first one.
The difference lies in the use of multiple neighbourhoods. When a local optima
is met, a larger neighbourhood is used. The program stops when a local optima is
met with the last neighbourhood available.\\

To do so, a \textbf{RulesFactory::getVNDNeighbourhood()} method was added. It
returns an array of 3 pointers to a Neighbourhood. The 3 Neighbourhoods are used
in the 2 following orders: transpose-exchange-insert or transpose-insert-exchange,
with transpose having a size of $(n-1)$ elements, exchange a size of
$n * \frac{n-1}{2}$ elements, and insert a size of $(n-1)^2$ elements.\\

\begin{lstlisting}
Neighbourhood** RulesFactory::getVNDNeighbourhood(std::string neighbourhood,
    unsigned size)
{
    Neighbourhood** n = new Neighbourhood*[3];
    Neighbourhood* nt = new TransposeNeighbourhood(size);
    Neighbourhood* ne = new ExchangeNeighbourhood(size);
    Neighbourhood* ni = new InsertNeighbourhood(size);
    
    if (neighbourhood == "tei") {
        n[0] = nt; n[1] = ne; n[2] = ni;
        return n;
    }
    else if (neighbourhood == "tie") {
        n[0] = nt; n[1] = ni; n[2] = ne;
        return n;
    }
    
    exit(EXIT_FAILURE);
}
\end{lstlisting}
\

The main body of the algorithm is contained in the
\emph{VariableNeighbourhoodDescent} class which can be seen below.\\

\begin{lstlisting}
void VariableNeighbourhoodDescent::run() {
    Permutation p1 = initialization_.generateInitialization();
    
    int i = 0;
    while (i < 3) {
        Permutation p2 = improvement_.improve(p1, *neighbourhoods_[i]);
        
        if (p1 == p2) {
            i++;
        }
        else {
            p1 = p2;
            i = 1;
        }
    }
}
\end{lstlisting}

\subsection{Results}

\begin{lstlisting}
experiment_name average_delta time_sum time_average

vnd_tei 2.6595640233508 1970.80062 25.266674615385
vnd_tie 2.6487053958858 1352.99196 17.346050769231
\end{lstlisting}

\newpage

\subsection{Tests}

\noindent Order in which the rows \& columns are ordered in the tables :

\begin{enumerate}
    \itemsep-0.5em
    \item vnd\_tei
    \item vnd\_tie
\end{enumerate}

\subsubsection{Student's t-test}

\begin{table}[h]
\begin{tabular}{|l||l|l|}
\hline
 & 1 & 2 \\ \hline
\hline
1 & 0 & 0.853 \\ \hline
2 & 0 & 0 \\ \hline
\end{tabular}
\end{table}

\subsubsection{Wilcoxon signed-rank test}

\begin{table}[h]
\begin{tabular}{|l||l|l|}
\hline
 & 1 & 2 \\ \hline
\hline
1 & 0 & 0.625 \\ \hline
2 & 0 & 0 \\ \hline
\end{tabular}
\end{table}

\newpage
